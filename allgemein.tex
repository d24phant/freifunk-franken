\section{Freifunk: die Idee}
\subsection{Über Freifunk}

\begin{frame}
\frametitle{Was ist Freifunk überhaupt?}
\begin{itemize}
\item \href:{https://vimeo.com/64814620}{Video: Freifunk verbindet!}
\item Aufbau eines vermaschten Netzwerkes -> Meshnetz
\item Demokratisierung einer Infrastruktur
\item Zur Verfügung stellen freier Kommunikation für alle Personen<-> Offenes WLAN
\item Netzneutralität
\item Unabhängige \textbf{unkommerzielle Infrastruktur} in Bürgerhand
\item Kostenlose, benutzerfreundliche \textbf{Hotspot-Gemeinschaft}
\end{itemize}
\end{frame}

\begin{frame}
\frametitle{Wer macht Freifunk?}
\begin{itemize}
\item \textbf{Hierachiefreie}, \textbf{offene}, überregionale Gemeinschaft, die diese Grundsätze verfolgen will
\item Personen, die \textbf{Router} aufstellen, \textbf{Firmware} schreiben oder \textbf{Server} bereitstellen
\item Beginnt schon damit, dass eine Person einen Router für andere Menschen zur Verfügung stellt
\item Unabhängige \textbf{unkommerzielle Infrastruktur} in Bürgerhand
\item Dezentral weitflächig über Deutschland verteilt
\end{itemize}
\end{frame}

\subsection{Wie funktioniert Freifunk?}

\begin{frame}
\frametitle{Wie funktioniert Freifunk?}
\begin{itemize}
\item WLAN-Mesh-Technologie “\textbf{Funk-Maschennetz}”
\item Diese Technik eignet sich besonders gut um geografische und soziale Lücken zu schließen.
\item \textbf{VPN vernetzt} „Inseln” und Städte
\item Billig, Robust und ausfallsicher
\item Für Alle immer frei und \textbf{unzensiert}
\end{itemize}
\end{frame}


\begin{frame}
\frametitle{}
\includegraphics[scale=0.3]{images/personal_setup.png}
\end{frame}



\begin{frame}
\frametitle{}
\begin{itemize}
\item 
\end{itemize}
\end{frame}
